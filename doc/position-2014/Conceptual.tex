\subsection{Conceptual Software Design}
We describe some of the primary elements of this new
conceptual model while addressing
the multitude of design challenges described above.
These models form the basis for the software design
of \flux.

\vspace{1ex}
\noindent{\em Unified Job Model:} Traditionally, a job is
simply defined to be a resource allocation, a concept
too weak to support the new paradigm. Rather, we unify
the traditional job notion with the notion of a resource
manager instance---an independent set of resource
manager services. The RM instance must be delegated
the main responsibility of managing the resources allocated
to the job. Then, the unified job model becomes
the foundation on which to build a hierarchical, 
resource-management scheme to address the multidimensional
scale challenge. In addition, an RM instance can
implement compatibility mode with a particular
traditional paradigm only over its own allocation, 
providing a straightforward path to address
the backward compatibility challenge.

\vspace{1ex}
\noindent{\em Job Hierarchy Model:} To scale the new conceptual model
in the scaling limit of the entire computing facility, 
we must avoid a centralized approach: the new paradigm 
requires a hierarchical management scheme with a well-balanced, 
multilevel delegation structure. For this purpose, 
we use a tree-based job hierarchy model that has 
many proven advantages for extreme scalability. 
In this model, a job is only required to manage 
its children jobs, which would be only a small fraction 
of the total number of jobs that are run across 
the entire computing facility. Further, several guiding 
principles throughout the job hierarchy strike 
a balance between the management responsibility 
of a parent job and delegation and empowerment 
of a child job:

\begin{enumerate}
\item{Parent bounding rule: the parent job grants 
and confines the resource allocation of all of its children.}

\item{Child empowerment rule: within the bound set 
by the parent, the child job is delegated the ownership 
of the allocation and becomes solely responsible 
for most efficient uses of the resources.}

\item{Parental consent rule: the child job must ask 
its parent job when it wants to grow or shrink the resource 
allocation, and it is up to the parent to grant the request.}
\end{enumerate}

In general, these rules enforce the first principle 
of the new model: imposing highly complex resource bounds 
to guarantee the highest operational efficiency 
at any level across the computing facility, while enabling 
most efficient execution and scheduling of the workloads 
within these bounds. 

At the same time, this model is the most fundamental 
design concept, which forms the basis to address 
many of the design challenges including 
the {\em multidimensional scale}, {\em rich resource model},
and {\em dynamic workload} challenges previously described.

\vspace{1ex}
\noindent{\em Generalized Resource Model:} In the traditional 
paradigm, compute resources are modeled primarily 
as a collection of compute nodes, a simplistic perspective 
ill-suited for the new paradigm. Today's applications 
are diverse with disparate limiting performance factors 
beyond floating point computation. 

Further, computing centers are increasingly concerned 
about managing new resource types such as power 
and shared persistent storage. The generalized resource 
model is our concept to represent various resource types 
and their relationships that can impact how well applications 
perform and the computing facility operates. 
Our generalized resource model also includes a unified 
resource specification and description language. 
Speaking the same resource description language 
for request specification provides transparency and 
fine-grained expressibility. Our generalized resource 
model addresses the {\em diverse workload} challenge.

More specifically, the unified language approach 
allows users to express their resource requests 
more flexibly, e.g., using ranges or boolean 
expressions instead of hard amounts to allow requests 
to be fulfilled from several equivalent resource types. 
This makes the scheduling granularity 
of jobs finer and more malleable.

\vspace{1ex}
\noindent{\em Resource Allocation Elasticity Model:} As our 
applications and their programming models are becoming 
increasingly dynamic, the new paradigm must support 
an elasticity model where an existing resource allocation 
can grow and shrink, depending on the current needs 
of applications and/or the computing facility. 

We support the elasticity model within our job hierarchy 
framework above: a child job sends a grow or shrink request 
to its parent, which can go up the job hierarchy 
until all requisite constraints are known for this request. 
Also, combining this with the generalized resource model, 
the elasticity can be expressed for any resource including 
power consumption. Our elasticity model addresses 
the {\em dynamic workload} challenge.

\vspace{1ex}
\noindent{\em Common Scalable Persistent Communication Infrastructure Model:} 
Our scalability strategy with respect to a large number 
of compute nodes is to provide a common scalable communication 
framework within each job. When a job is created, a secure, scalable 
overlay network with common communication service is established 
across its allocated nodes. Except for the root-level job, 
the existing communication session of the parent job assists 
the child job with rapid creation of its own session. 

A communication session is only aware of its parent 
and child and passes the limited set of control information 
through this communication channel. Thus, this model 
enables highly scalable communication within a job, while 
limiting communications between jobs, addressing 
both the {\em multidimensional scale} as well as security issues.

Further, this backbone per-job communication network 
supports many well-known bootstrap interfaces 
for distributed programs including many MPI implementations 
as well as run-time tools.  This provides tightly integrated support
for the development and use of scalable code development run-time tools
and research, which can have a large impact on user productivity.
