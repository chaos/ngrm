It is a serious undertaking 
to design the production version of a new 
RM framework that can realize our proposed conceptual models.
Thus, \flux\ team has engaged in a series of steps
meant to inform our design process. 
We began with a functional design, reviewed
by a small group of stakeholders including 
systems, run-time, and applications developers,
as well as data center management.
Subsequently, we initiated a prototyping phase for the \flux\ run-time.


%Then, this notion provided us with a concise 
%mechanism by which we can relate groups of processes
%to underlying resources (e.g., containing certain processes 
%to a set of resesources) as well as relate groups
%one another (e.g., synchronizing tool processes to 
%MPI processes). 
%
As part of the protyping exercise, we have developed a communication framework 
called the Communication Message Broker (CMB) 
as well as a scalable key value store (KVS) service.
They comprise essential scalable building blocks of our run-time system, 
and have allowed us to build various run-time services as well as
to port user-level run-time software programs to our prototype.
Further, we developed support for our concept of the lightweight 
job (LWJ) as a new way to organize distributed or parallel processes.
Using this notion, a wide range of run-time software
programs such as parallel programming models,
tools, and middleware can leverage our common run-time services 
to build upon \flux\ and each other.

\ifcomments
\marginpar{\tiny BS: Do you want to now make a case for why KVS is going
to be an important building block?  Or just launch into the next
two sections on CMB and V'S?}
\marginpar{\tiny DA: I think I made a case. Let me know if not}
\fi
