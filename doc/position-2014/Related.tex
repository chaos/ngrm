\section{Related Work}

Wang et al propose using a distributed key-value store
as the basis for HPC tools and services in order to encapsulate
distributed complexity in the KVS, thereby simplifying the
tools and services~\cite{Wang:2013:USE:2503210.2503239}.
They further evaluate replacing the centralized controller in
Slurm~\cite{Jette02slurm} with a distributed controller~\cite{Slurmpp}
built on the ZHT distributed hash table~\cite{Li:2013:ZLR:2510661.2511401}.

Redis~\cite{Redis} and twemproxy~\cite{Twemproxy}
were employed in early \flux\ KVS investigations.
Redis cluster~\cite{RedisClusterTut,RedisClusterSpec} is now available.
Redis cluster nodes establish TCP connections in a full mesh, and implement
sharding, re-sharding, replication, and failover.
There is no proxy; clients connect directly to the server
for a particular shard.  Redis-cluster's asynchoronous replication
results in the possibility of losing writes that occur during failover.

Arnold and Miller's MrNet~\cite{mrnet} is a reusable tree-based overlay
network (TBON) designed to be embedded in large scale HPC tools such as
STAT~\cite{STAT}.  They have explored
{\em state compensation}~\cite{conf/ipps/ArnoldM10}
as a mechanism for fault tolerance in TBONs that perform data aggregation.
