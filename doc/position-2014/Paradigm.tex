\section{A New Conceptual Model for HPC Resource Management}
The vision of \flux\ is to create a scalable RM
software system that significantly improves
operational efficiency and user productivity
for resources spanning an entire
computing facility.
Our challenges include providing extreme
scalability, low noise, and fault tolerance while
managing arbitrarily complex aggregations of generic resources
which may be subject to multiple dimensions of constraints
such as power budgets, IO throughput bounds, and other site-specific
policy.
Worse, the workloads themselves continue to become
increasingly diverse, dynamic, and large.
Thus, fully realizing our vision through
these challenges requires a major shift
in the conceptual model of how the RM must manage, model, schedule,
and allocate its resources.

\subsection{Design Challenges}
\label{sect:challenges}

The first requirement in designing \flux\ based on this
new conceptual model is to take a center-wide, or global,
resource view.  The RM must be capable of
imposing highly complex resource bounds
to guarantee the highest operational efficiency
at any level across the computing facility, 
while at the same time enabling the most efficient
execution and scheduling of the workloads
within these bounds.
This requires the RM to have purview over 
the entire computing facility. 
The RM must manage the resources at the center
as one common pool of resources, and the ability
to see a broader spectrum of resources
and their various constraints can then enable
more efficient scheduling strategies
and execution environments. 

We characterize this design challenge as the {\em multidimensional
scale} challenge.  Simply stated, the challenges include
supporting extreme scalability, addressing noise as concurrency
increases, and managing a drastically increased amount of
run-time information that must be monitored, traced, and stored.
Our requirement is that our RM shall handle increased scale in
numbers of resources as well as jobs and other dimensions of RM data.

A second requirement for \flux\ is to include a rich resource model.
This includes a generalized model of diverse types of resources,
such as file systems, visualization systems, and serial batch systems,
in addition to traditional clusters.   With a richer resource model,
the RM will be capable of imposing complex resource bounds,
as opposed to the traditional view of CPUs, memory, and time limits.
With a \flux\ RM that has the ability to model general resource
relationships, as opposed to flat lists of nodes, we will be able
to allocate computing resources tailored to the disparate limiting
factors of our applications.  For example, an application may be
compute bound while others are I/O bound or power bound.
This approach will enable stronger efforts to diagnose errors
for both end users and support staff by associating jobs
with other facility-wide events.

We chacterize this design challenge as the {\em diverse workload
challenge}.   As one specific example of emerging resource types,
power is becoming a critical factor. When the computing facility
becomes power bound instead of compute-node bound, the RM design
must enable the scheduling of workloads based upon the maximum
power limit at any level of the facility. Thus the resource
representation must be generalized enough to model consumable
resources like power, as well as the diversity of hardware.

A third requirement is that resource allocations
must also be elastic. An application may have different
phases with disparate performance-limiting factors;
it must be able to grow and shrink its resource allocation
dynamically.  This is in contrast to traditional methods with
static time-limit-bounded allocations.

We characterize this design challenge as the {\em dynamic workload
challenge}.  
Not only must the paradigm support disparate performance limiters
across different applications, but also must it suit varying
performance limiters within a single application. Our applications
and their programming paradigm are becoming increasingly dynamic with
different resource requirements at different phases.

Finally, the new conceptual model must meet the greater difficulties
in code development by facilitating the integration 
of other key relevant software that can ease the difficulties. 
These software components should include system monitoring
and administration, lightweight virtualization,
and scalable tool communication.
The integration will facilitate a higher level of
leverage among these essential computing elements,
and this will lead to significantly higher productivity
for both end users and system administrators.
These capabilities are currently provided
through disjoint and often overlapping software,
the integration of which can provide a richer, stronger environment
and may reduce development costs through the ability to leverage
the \flux\ framework.
We characterize this design challenge as the {\em productivity challenge}.

We have considered additional challenges that we
have yet to explore fully. For example,
another challenge that we considered during the design included
the risk of higher downtime costs in a more global model.
If the RM is not designed adequately, a downtime could negatively
impact the availability of a large portion of the HPC center’s
resources. Thus our RM must be tolerant of hardware and software
faults and failures with no single point of failure and must
also support live software upgrades. This and other challenges,
including security, integration risk, and backwards compatibility will be 
will be addressed in future prototypes.

In summary, the global resource view, rich resource model,
elasticity, and seamless integration of other software 
represent the fundamental characteristics of the new
resource management paradigm.  

