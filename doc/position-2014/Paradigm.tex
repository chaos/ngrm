\section{A New Paradigm for HPC Resource Management}
The vision of \flux\ is to create a scalable RM
software system that significantly improves
operational efficiency and user productivity
for resources spanning an entire
computing facility.
Our challenges include providing extreme
scalability, low noise, and fault tolerance while
managing arbitrarily complex aggregations of generic resources
which may be subject to multiple dimensions of constraints
such as power budgets, I/O throughput bounds, and other site-specific
policy.
In addition, the workloads themselves continue to become
increasingly diverse, dynamic, and large.
To address these challenges, the \flux\ design must include
new conceptual models for resource management
that comprise a new paradigm.

\subsection{Design Challenges}
\label{sect:challenges}

The first requirement in designing \flux\ to 
embody the new paradigm is that it must impose
complex, multidimensional resource bounds at
any scale, from the center-wide or global level,
down to the level of individual processes,
and enable the most efficient execution and
scheduling of workloads within these bounds.
Only with RM purview over the entire computing
facility can resources be managed at optimal operational efficiency.

We characterize this design challenge as the {\em multidimensional
scale} challenge.  Simply stated, the challenges include
supporting extreme scalability, addressing noise as concurrency
increases, and managing a drastically increased amount of
run-time information that must be monitored, traced, and stored.
The requirement is that our RM shall handle increased scale in
numbers of resources as well as jobs and other dimensions of RM data.

A second requirement for \flux\ is to include a rich resource model.
This includes a generalized model and representation for diverse
types of resources such as file systems, networks, visualization
hardware, and heterogeneous compute engines, in addition to
traditional clusters.
With a richer resource model, the RM will be capable of imposing
complex, multidimensional resource bounds, as opposed to the
simplistic traditional resource model that is fundamentally based on
flat lists of nodes.
With the ability to model sophisticated resource relationships,
\flux\ will be able to schedule and
allocate resources tailored to the disparate limiting
factors of our applications.  For example, an application may be
compute bound while others are I/O bound or power bound.
This approach will enable stronger efforts to diagnose errors
for both end users and support staff by associating jobs
with other facility-wide events.

We characterize this design challenge as the {\em diverse workload
challenge}.   As one specific example of emerging resource types,
power is becoming a critical factor. When the computing facility
becomes power bound instead of compute-node bound, the RM design
must enable the scheduling of workloads based upon the maximum
power limit at any level of the facility. Thus the resource
representation must be generalized enough to model continuously consumable
resources like power, as well as the diversity of hardware.

A third requirement is that resource allocations
must be elastic. An application may have different
phases with disparate performance-limiting factors;
it must be able to grow and shrink its resource allocation
dynamically.  This is in contrast to traditional methods with
static time-limit-bounded allocations.

We characterize this design challenge as the {\em dynamic workload
challenge}.  
Not only must the paradigm support disparate performance limiters
across different applications, but also must support dynamically
changing performance limiters within a single application.  HPC applications
and their programming paradigm are becoming increasingly dynamic with
different resource requirements at different phases.

Finally, the new paradigm must address increasing complexities
in code development and system administration by facilitating
the creation of more effective diagnostic and analysis tools.
By providing basic, scalable monitoring and communication
primitives at the job level that can be leveraged by tools,
\flux\ will encourage a richer, stronger tool ecosystem where
these basic but non-trivial capabilities need not be recreated
from scratch in every tool, reducing development costs.
Better tools will lead to lead to higher productivity for all
stakeholders, including end users.
We characterize this design challenge as the {\em productivity challenge}.

We have considered additional challenges that we
have yet to explore fully. For example,
another challenge that we considered during the design included
the risk of higher downtime costs in a more global model.
If the RM is not designed adequately, a downtime could negatively
impact the availability of a large portion of the HPC center’s
resources. Thus our RM must be tolerant of hardware and software
faults and failures with no single point of failure and must
also support live software upgrades. This and other challenges,
including security, integration risk, and backwards compatibility will be 
will be addressed in our future work.

%In summary, the global resource view, rich resource model,
%elasticity, and seamless integration of other software 
%represent the fundamental characteristics of the new
%resource management paradigm.  
%
