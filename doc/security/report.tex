\documentclass[10pt]{article}

\usepackage{verbatim}
\usepackage{calc}
\usepackage{epsfig}
\usepackage{url}
\usepackage{longtable}
\input{pstricks}
\usepackage{multirow}
\usepackage{epstopdf}
\usepackage{graphicx}
\usepackage{type1cm}
\usepackage{eso-pic}
\usepackage{color}
\usepackage{cite}
\usepackage{listings}

\usepackage{subfig}

% draft watermark begin
\makeatletter
\AddToShipoutPicture{
	\setlength{\@tempdimb}{.5\paperwidth}
	\setlength{\@tempdimc}{.5\paperheight}
	\setlength{\unitlength}{1pt}
	\put(\strip@pt\@tempdimb,\strip@pt\@tempdimc){
		\makebox(0,0){\rotatebox{55}{\textcolor[gray]{0.85}
			{\fontsize{5cm}{5cm}\selectfont{DRAFT}}}}
	}
}
\makeatother
% draft watermark end

\graphicspath{ {./fig/} }

\setlength{\textwidth}{6in}
\setlength{\textheight}{9in}
\setlength{\oddsidemargin}{(\paperwidth-\textwidth)/2 - 1in}
\setlength{\topmargin}{(\paperheight-\textheight -\headheight-\headsep-\footskip)/2 - 1.2in }

% enable/disable WBS tables
\newif\ifwbs
\wbsfalse
%\wbstrue

% enable/disable margin comments
\newif\ifcomments
%\commentsfalse
\commentstrue

\newcommand{\flux}{Flux}
\newcommand{\fluxfull}{Flux Resource Manager}
\newcommand{\zMQ}{\O{}MQ}
\newcommand{\slurm}{Slurm}
\newcommand{\moab}{Moab}

\DeclareRobustCommand{\orderof}{\ensuremath{\mathcal{O}}}

\begin{document}

\title{\fluxfull\ Security Architecture\\
{\large LLNL-TR-XXXXXX-DRAFT}}
\author{\
Jim Garlick, garlick@llnl.gov}

%\date{April 1, 2014}

\maketitle

\section{Introduction}

This document is for \flux\ developers and those helping us review and
harden \flux\ security.
Section~\ref{sec:background} reminds us broadly what problems need
to be solved by the \flux\ security architecture.
Section~\ref{sec:softarch} then describes how \flux\ components
interact.
A detailed security proposal is presented in Section~\ref{sec:security}.
It is tested against various use cases in Section~\ref{sec:usecases}.

\section{Background}
\label{sec:background}

\flux\ is middleware that facilitates remote execution across a site.
It must authenticate
and authorize users to resources and resources to users according to
site policy.  It must protect resources from inappropiate use,
data from inappropriate accesss, and users from being interfered with
by other users and outsiders.

\flux\ is presumed to be deployed within a single organization that
manages users and services in a coordinated way, and limits outside
access in a manner appropriate for the site's security sensitivity.
\flux\ is not expected to provide services on the public Internet,
nor to implement grid proxy security.

On the other hand, \flux\ is intended to be deployed more broadly
across a site than cluster-centric resource managers, and may
require communication over networks that are less physically secure
than the private networks within a cluster.

\section{Job Launch Mechanism}
\label{sec:softarch}

A key design attribute of \flux\ is that a job is a full resource
manager instance that is capable of spawning additional jobs.
Every job is assigned some set of resources and has a {\em job owner}.
We define the {\em init job} to be the first job bootstrapped on
a resource set.  The job owner of the init job will be
a unique, unprivileged pseudo-user called the {\em resource owner}.

Integral to each job is a {\em comms session}, an instance of the \flux\ 
communications framework consisting of one or more interconnected
{\tt flux-cmbd} comms message broker daemons which run as the job owner.
\flux\ resource manager services are implemented as comms modules,
plugins that are loaded into {\tt flux-cmbd} under control of
the job owner.  There is no mechanism to protect the comms session from
a malicious job owner---this part of the system is designed to be configurable
and even replaceable by the job owner.

This flexibility is made possible by the fact that each
\flux\ job runs within a set of containers created and managed its
parent.  The containers serve dual purposes:
limiting resource consumption to bounds set by the parent,
and providing a uniform interface to software running at any level
of job recursion.
Each container is managed by a privileged process, {\tt flux-launchd},
which provides process launch capability, while enforcing site policy
and parental resource bounds.
{\tt flux-launchd} is the process-ancestor of all processes in the container,
including {\tt flux-cmbd}.  Both {\tt flux-launchd} and {\tt flux-cmbd}
listen on UNIX domain sockets at well-known paths within the container.
A \flux\ container is depicted in Figure~\ref{fig:container}.
\begin{figure}
\centering
  \includegraphics[scale=0.5]{container}
\label{fig:container}
\caption{Standard \flux\ container.}
\end{figure}

The init job is launched automatically on each node when it boots.
Since it has no parent, the init job's containers are statically
configured, beginning with {\tt flux-launchd} startup via {\tt init}(8)
or equivalent.
{\tt flux-launchd} then starts {\tt flux-cmbd}, which connects to
its statically configured overlay peers.
Figure~\ref{fig:launch}(a) shows an init job, bootstrapped.

\begin{figure}
\centering
\subfloat[The init job starts in its ``container''.
{\tt flux-launchd} (L) uses {\tt flux-launchd-helper} to start
{\tt flux-cmbd} as the resource owner.
The {\tt flux-cmbd} processes on each node attach to overlay network peers.]{{
  \includegraphics[width=6cm]{launch1} }}
\qquad
\subfloat[The init job directly starts an LWJ.
{\tt flux-cmbd} launches LWJ processes (P) using services of
{\tt flux-launchd} (L).  {\tt flux-lwj-helper} manages stdio and signals for P,
using services of {\tt flux-cmbd}.]{{
  \includegraphics[width=6cm]{launch2} }}
\caption{\flux\ init job on three nodes of cluster {\em atlas}.
Dotted lines are process parent relationships,
solid arrowed lines indicate communication.
Processes running as root are colored red.}
\label{fig:launch}
\end{figure}

The init job executes distributed work
by arranging to call out in parallel from a subset of its
{\tt flux-cmbd} processes to {\tt flux-launchd}, which starts
new processes in the container on its behalf, possibly as a different user.
%The details of this environment are arbitrated by {\tt flux-launchd},
%based on input from its configuration, the user's job request,
%and the init job's run directive.
Figure~\ref{fig:launch}(b) shows work being launched directly by the
init job.  We distinguish {\em job} from {\em lightweight job} (LWJ)
by whether or not a new container with a new comms session is created.
In the case of an LWJ, work is directly spawned as shown in the figure.
If the LWJ were an MPI job requiring PMI services, the \flux\ PMI
implementation would utilize the init job's KVS.

A full job launch is similar to the LWJ launch, except that a new
sub-container is instantiated, and a new comms session within it.
From the perspective of the parent job, the child job looks similar to
a LWJ, for example, stdio and signals for the child {\tt flux-cmbd} processes
are managed by the parent and its launch information is obtained from
the parent KVS.

\flux\ utilities interact with the comms session using either a UNIX
domain ``API socket'' (if running in the same container as a job)
or via an external TCP socket.  Messages sent via these mechanisms
have to be stamped with the user id of the sender. Comms modules
processing these messages, for example the KVS module processing
get and put requests, have to compare the sender against a
{\em capability list} for the job.  The job owner has the complete
set of capabilities, and other users are granted access and capabilities
by the job owner.

\section{Proposed Security Architecture}
\label{sec:security}

The \flux\ communications framework overlay network
is built upon the \zMQ\cite{ZMQGuide} messaging library.
\zMQ\ implements privacy, integrity, and authentication
between network peers using public key encryption (PKE).
Leaving aside the details of \zMQ's PKE implementation for now,
the general idea is that each user generates a public, private keypair.
A \zMQ\ endpoint must be configured with this keypair and have knowledge
of the public keys of any users with which it will communicate.
After an initial handshake, connection of endpoints only proceeds
if the private key used to connect matches one of the public
keys known on the other end and vice-versa.  Once the handshake is
complete, privacy and integrity are enabled on the channel
between those endpoints.

This security capability is used to secure the endpoints
used in a \flux\ comms session's overlay network.  The job owner's
keypair is configured on all endpoints.  Connection by other users
is not permitted and messages exchanged within the comms session have
privacy and integrity.  Thus the the first aspect of our security
architecture is that users need to have \flux\ keypairs.

In order to bootstrap the init job's comms session, each {\tt flux-cmbd}
needs to be launched with knowledge of the resource owner's keypair.
This is the point at which security is bootstrapped in the system.
The resource owner's keypair is provided out of band in the
static configuration of {\tt flux-launchd} on that resource set
(say a cluster), so that the init job's {\tt flux-cmbd} processes can be
launched and begin communicating securely.  Generating the resource
owner's keypair and distributing it is one of the tasks a
system administrator would complete to configure \flux\ on a new system.

When a user requests to run an LWJ within the init job, they first
authenticate to the session, either via the API socket with {\tt SO\_PEERCRED}
or externally using some method to be defined (such as Kerberos).
A comms module handling such requests compares the requestor identity
stamped on the request message against the init job's capability list.
If the user has the appropriate capability, and other conditions such as
scheduling are met, a scalable launch mechanism in the style of

is initiated, except that the launch of {\tt wrexecd} and its children
is arranged via {\tt flux-launchd} to allow privileged operations such as
setting the effective uid to be performed as root.

When a user wants to run a job within the init job, the user's keypair
will be needed so that a comms session can be started as that user and
communicate securely.

\section{Use Cases}
\label{sec:usecases}

\cite{Flux}

\bibliographystyle{abbrv}
\bibliography{../bib/project,../bib/rfc}

\end{document}
